\documentclass[oneside, 12pt]{report}
\usepackage[utf8]{inputenc}

\usepackage{amsmath}
\usepackage{float}
\usepackage{amssymb}
\usepackage{indentfirst}
\usepackage{pythonhighlight}
\usepackage{framed}
\colorlet{shadecolor}{blue!20}

\usepackage{hyperref}
\hypersetup{
    colorlinks=true,
    linkcolor=blue,
    filecolor=magenta,      
    urlcolor=cyan,
}
\urlstyle{same}

\usepackage[a4paper, total={7.5in, 10.7in}]{geometry}
\setlength{\parskip}{0.5em}
\setlength{\parindent}{0pt}

\title{NumPy As Strided Primer}
\date{2020\\ December}
\author{John Chang}

\begin{document}
\maketitle
\newpage

\chapter{NumPy As Strided Primer}

\section{Motivation}

\verb+as_strided+ is one of the core reasons why NumPy is fast.

A use case would be getting $2 \times 2$ windows from a $10 \times 10$ array as a reference using only a single call.

References are efficient because \textbf{read-only} operations do not require copying values to another memory address.

\section{Introduction}

The assumption that the arrays are continuous is essential for fast memory addressing and retrieval.

For example, 3 numbers next to each other in memory.

$$(10, 20, 30)$$

If they are represented by 1 Byte, in an unsigned fashion, we can achieve a range of $[0, 2^{8}-1] = [0, 255]$. Assume that in this context, they only need to be Byte aligned.

This makes it easy for NumPy to dispatch a data retrieval as they are partitioned evenly, getting the next value is just $address + itemsize$. The following illustrates the memory used for storing the values shown above.

$$(0A_{16}, 14_{16}, 1E_{16})$$

NumPy only then needs to know the following:

\begin{itemize}
\item{Where does this address start?}
\item{How large is each object?}
\item{How many objects are there?}
\end{itemize}

We're not too concerned about start addresses, they are usually handled in the back-end. However, we can redefine object size, length, and stride.

\section{1D Striding}

Redefining strides is done using \verb+np.lib.stride_tricks.as_stride+

There are 2 main arguments we're concerned about.

\begin{itemize}
\item{Stride}
\item{Shape}
\end{itemize}

Data is \textbf{always} stored as a 1-Dimensional array in memory.

\subsection{Stride By 1}

E.g. the following memory chunk, $n \in [0, 255]$

\begin{verbatim}
VALUE 000 010 020 030 040 050 100 110 120 130 140 150
----- ----------------------------------------------- 
BYTE  005 006 007 008 009 010 011 012 013 014 015 016
\end{verbatim}

To get \textbf{all values}, we'd stride by a single byte, that is, we'd retrieve \verb+005, 006, 007, ...+. It's analogous to the step size, take a single byte-sized step every time.

This can be done in the following code. Note that this is incomplete due to the missing \verb+shape+ argument but it will still run.

\begin{python}
arr = np.append(np.arange(0, 60, 10, dtype=np.uint8),
                np.arange(100, 160, 10, dtype=np.uint8))
np.lib.stride_tricks.as_strided(arr, strides=(1,))
\end{python}

A few things to note:
\begin{enumerate}
\item{\verb+np.uint8+ makes each value 1 byte large, what happens if you don't use this?}
\item{\verb+strides+ must be a \verb+Iterable+, that's why it's \verb+(1,)+ and not \verb+(1)+}
\end{enumerate}

What do you expect as the output? Logically it's going to be the same array, and it is.

\begin{verbatim}
>> array([  0,  10,  20,  30,  40,  50, 100, 110, 120, 130, 140, 150],
     	 dtype=uint8)
\end{verbatim}

\subsection{Striding By N}

It should be intuitive what happens when you stride by N but let's go through to sanity check.

\begin{verbatim}
VALUE 000 010 020 030 040 050 100 110 120 130 140 150
----- ----------------------------------------------- 
BYTE  005 006 007 008 009 010 011 012 013 014 015 016
       +       +       +       +       +       +
      ptr
\end{verbatim}

Note that when you create an array, there's a pointer \verb+ptr+ that indicates where the array starts. So when you do strides, it's relative to that pointer.

\begin{python}
arr = np.append(np.arange(0, 60, 10, dtype=np.uint8),
                np.arange(100, 160, 10, dtype=np.uint8))

np.lib.stride_tricks.as_strided(arr, strides=(2,))
\end{python}

\begin{verbatim}
>> array([  0,  20,  40, 100, 120, 140,  28,   0,   0,   0,   0,   0],
   	     dtype=uint8)
\end{verbatim}

Interestingly enough, you may not even get the same result as I did, what happened here was an \textbf{Out of Bounds} memory access.

Remember an argument called \verb+shape+, the shape is implicitly passed thus algorithm assumes the \textbf{same shape as the input}. That's why it went beyond and out of bounds.

As warned in the NumPy Documentation on \href{https://docs.scipy.org/doc/numpy-1.15.1/reference/generated/numpy.lib.stride_tricks.as_strided.html}{as\_strided}, accessing Out of Bounds is fine, but writing on it may corrupt memory.

For this practise \verb+writeable+ by default, is \verb+False+, therefore it won't run into any corruption problems but it's good to note.

\begin{verbatim}
VALUE 000 010 020 030 040 050 100 110 120 130 140 150 ??? ??? ??? 
----- ----------------------------------------------------------- ...
BYTE  005 006 007 008 009 010 011 012 013 014 015 016 017 018 019
       +       +       +       +       +       +       +       +
\end{verbatim}

\subsection{Shape}

Shape is the bounds of where your strides should end at.

By halving and rounding the shape, we limit where the stride should end. This requires the input shape, so we grab the \verb+shape+ property.

\begin{python}
np.lib.stride_tricks.as_strided(arr, strides=(2,), shape=(a.shape[0]//2,))
\end{python}

Note the extra comma to cast it as an \verb+Iterable+.

\begin{verbatim}
>> array([  0,  20,  40, 100, 120, 140, 160], dtype=uint8)
\end{verbatim}

\section{2D Striding}

We're going to wrap a 1D memory as a 2D object

Take for example the following

\begin{verbatim}
VALUE 000 001 002 010 011 012 020 021 022 030 031 032
----- -----------------------------------------------
BYTE  001 002 003 004 005 006 007 008 009 010 011 012
\end{verbatim}

The end result we want is to wrap it such that we have

\begin{equation*}
\begin{bmatrix}
0  & 1  & 2 \\
10 & 11 & 12 \\
20 & 21 & 22 \\
30 & 31 & 32
\end{bmatrix}
\end{equation*}

In Python, it'll be shown as

\begin{verbatim}
[[0, 1, 2], [10, 11, 12], [20, 21, 22], [30, 31, 32]]
\end{verbatim}

Let's go by logic on how you would construct this matrix when given the 1D representation

\subsection{Understanding Multidimensional Striding}

Since you know that each row has 3 values, we take the first 3 values

\begin{verbatim}
VALUE 000 001 002
----- -----------
BYTE  001 002 003
\end{verbatim}

Then file them under $R_0$.

\begin{verbatim}
VALUE 010 011 012
----- -----------
BYTE  004 005 006
\end{verbatim}

Take the next, file them under $R_1$.

\begin{verbatim}
VALUE 000 001 002 | 010 011 012 | 020 021 022 | 030 031 032
----- ----------- | ----------- | ----------- | -----------
BYTE  001 002 003 | 004 005 006 | 007 008 009 | 010 011 012
          R_0           R_1           R_2           R_3
\end{verbatim}

Let's look at the following illustration, this is \textbf{important} to understand for the next few sections.

\begin{figure}[H]
\begin{equation*}
\begin{bmatrix}
001 & \xrightarrow{1B} & 002 &\xrightarrow{1B} & 003 \\
3B \downarrow \\
004 & \xrightarrow{1B} & 005 &\xrightarrow{1B} & 006 \\
3B \downarrow \\
007 & \xrightarrow{1B} & 008 &\xrightarrow{1B} & 009 \\
3B \downarrow \\
010 & \xrightarrow{1B} & 011 &\xrightarrow{1B} & 012 \\
\end{bmatrix}
\end{equation*}
\caption{Memory addresses of the selected elements}
\end{figure}

\begin{shaded}
Here, the horizontal elements represent Dim 0, while the vertical elements represent Dim 1.

Dim 1 Strides by 3, while Dim 0 Strides by 1.
\end{shaded}

\begin{python}
arr = np.asarray([0, 1, 2,
                  10, 11, 12,
                  20, 21, 22,
                  30, 31, 32], dtype=np.uint8)
 
np.lib.stride_tricks.as_strided(arr, strides=(3,1), shape=(4,3))
 
>> array([[ 0,  1,  2],
          [10, 11, 12],
          [20, 21, 22],
          [30, 31, 32]], dtype=uint8)
\end{python}

\subsection{Skipping Values}

This is not too interesting as it seems to be logical on how it works, but what happens if you stride a bit further?

\begin{figure}[H]
\begin{verbatim}
VALUE 000 001 002 010 011 012 020 021 022 030 031 032 ??? ??? ???
----- -----------------------------------------------------------
BYTE  001 002 003 004 005 006 007 008 009 010 011 012 013 014 015
\end{verbatim}
\end{figure}

\begin{figure}[H]
\begin{equation*}
\begin{bmatrix}
001 & \xrightarrow{1B} & 002 &\xrightarrow{1B} & 003 \\
4B \downarrow \\
005 & \xrightarrow{1B} & 006 &\xrightarrow{1B} & 007 \\
4B \downarrow \\
009 & \xrightarrow{1B} & 010 &\xrightarrow{1B} & 011 \\
4B \downarrow \\
013 & \xrightarrow{1B} & 014 &\xrightarrow{1B} & 015 \\
\end{bmatrix}
\end{equation*}
\caption{Memory addresses of the selected elements}
\end{figure}

\begin{shaded}
Dim 1 Strides by 4, while Dim 0 Strides by 1.

Note how $013, 014, 015$ are all undefined in this context!
\end{shaded}

\begin{python}
np.lib.stride_tricks.as_strided(arr, strides=(4,1), shape=(4,3))

>> array([[ 0,  1,  2],
          [11, 12, 20],
          [22, 30, 31],
          [28,  2,  0]], dtype=uint8)
\end{python}

As expected, we get random values at the end, however, we managed to \textbf{skip} values during striding.

\subsection{Overlapping Values}

What happens if you do too short of a stride?

\begin{figure}[H]
\begin{verbatim}
VALUE 000 001 002 010 011 012 020 021 022 030 031 032
----- -----------------------------------------------
BYTE  001 002 003 004 005 006 007 008 009 010 011 012
\end{verbatim}
\end{figure}

\begin{figure}[H]
\begin{equation*}
\begin{bmatrix}
001 & \xrightarrow{1B} & 002 &\xrightarrow{1B} & 003 \\
2B \downarrow \\
003 & \xrightarrow{1B} & 004 &\xrightarrow{1B} & 005 \\
2B \downarrow \\
005 & \xrightarrow{1B} & 006 &\xrightarrow{1B} & 007 \\
2B \downarrow \\
007 & \xrightarrow{1B} & 008 &\xrightarrow{1B} & 009 \\
\end{bmatrix}
\end{equation*}
\caption{Memory addresses of the selected elements}
\end{figure}

\begin{shaded}
Dim 1 Strides by 2, while Dim 0 Strides by 1.

Note how we can repeat the addresses, this is not dangerous as read-only data.
\end{shaded}

\begin{python}
np.lib.stride_tricks.as_strided(arr, strides=(2,1), shape=(4,3))

>> array([[ 0,  1,  2],
          [ 2, 10, 11],
          [11, 12, 20],
          [20, 21, 22]], dtype=uint8)
\end{python}

Interestingly enough, we got overlapping values, we can take it 1 step further by doing a \verb+(1,1)+ stride.

\begin{figure}[H]
\begin{verbatim}
VALUE 000 001 002 010 011 012 020 021 022 030 031 032
----- -----------------------------------------------
BYTE  001 002 003 004 005 006 007 008 009 010 011 012
\end{verbatim}
\end{figure}

\begin{figure}[H]
\begin{equation*}
\begin{bmatrix}
001 & \xrightarrow{1B} & 002 &\xrightarrow{1B} & 003 \\
1B \downarrow \\
002 & \xrightarrow{1B} & 003 &\xrightarrow{1B} & 004 \\
1B \downarrow \\
003 & \xrightarrow{1B} & 004 &\xrightarrow{1B} & 005 \\
1B \downarrow \\
004 & \xrightarrow{1B} & 005 &\xrightarrow{1B} & 006 \\
\end{bmatrix}
\end{equation*}
\caption{Memory addresses of the selected elements}
\end{figure}

\begin{shaded}
Dim 1 Strides by 1, and Dim 0 Strides by 1.
\end{shaded}

\begin{python}
np.lib.stride_tricks.as_strided(arr, strides=(1,1), shape=(4,3))

>> array([[ 0,  1,  2],
          [ 1,  2, 10],
          [ 2, 10, 11],
          [10, 11, 12]], dtype=uint8)
\end{python}

This is extremely useful if you want to create a sliding window.

Before we discuss about 3D Striding, let's take a look at the underlying representation of some arrays.

\section{Creation of Arrays}

When we create arrays, they are always 1-D under the fancy NumPy Interface.

To illustrate, when you do \verb+np.asarray([[3,4],[7,8]], dtype=np.uint8)+ you get the following in the memory.

\begin{verbatim}
VALUE  3   4   7   8
----- ---------------
BYTE   1   2   3   4
\end{verbatim}

\textit{Note that the Byte Address doesn't necessarily have to start from 1, it's for simplicity}

What NumPy does under the hood is that it creates specific strides and shapes such that it is interpreted as \verb+[[3,4],[7,8]]+

\begin{python}
arr = np.asarray([[3,4],[7,8]], dtype=np.uint8)
arr.strides
>> (2, 1)
arr.shape
>> (2, 2)
\end{python}

This is so that you don't have to deal with wrapping the array, and this depicts why NumPy is so fast in transformation such as \verb+reshape()+

\section{Value Size}

Notice how I always declared values as \verb+np.uint8+, this is because it's \textbf{exactly 1 Byte}. This simplifies the examples, let's take a look what happens if you have a \verb+dtype+ of \verb+np.uint16+.

\begin{python}
arr = np.asarray([[0, 1], [2, 3]], dtype=np.uint16)

arr.strides
>> (4, 2)

arr.shape
>> (2, 2)
\end{python}

Compared to \verb+np.uint8+

\begin{python}
arr = np.asarray([[0, 1], [2, 3]], dtype=np.uint8)

arr.strides
>> (2, 1)

arr.shape
>> (2, 2)
\end{python}

Notice how the stride is doubled, this is because \verb+np.uint16+ is double the size of \verb+np.uint8+.

\textbf{np.uint8 memory storage}
\begin{verbatim}
VALUE  0   1   2   3
----- ---------------
BYTE   1   2   3   4
\end{verbatim}

\textbf{np.uint16 memory storage}
\begin{verbatim}
VALUE  0       1       2       3   
----- ------------------------------
BYTE   1   2   3   4   5   6   7   8
\end{verbatim}

Since it takes 2 bytes, the stride must be larger. This difference is important when defining our \verb+as_strided+ as it doesn't factor in the size of each value.

Stride too small or oddly and you'll get weird values

\textbf{Too small of a Stride}
\begin{python}
arr = np.asarray([[0, 1], [2, 3]], dtype=np.uint16)
np.lib.stride_tricks.as_strided(arr, strides=(2, 1), shape=(2, 2))

>> array([[  0, 256],
          [  1, 512]], dtype=uint16)
\end{python}

\begin{figure}[H]
\begin{verbatim}
VALUE  0       1       2       3   
----- ------------------------------
BYTE   1   2   3   4   5   6   7   8
\end{verbatim}
\end{figure}

\begin{figure}[H]
\begin{equation*}
\begin{bmatrix}
1 & \xrightarrow{1B} & 2 \\
2B \downarrow \\
3 & \xrightarrow{1B} & 4 \\
\end{bmatrix}
\end{equation*}
\end{figure}

\textbf{Too large of a Stride}
\begin{python}
arr = np.asarray([[0, 1], [2, 3]], dtype=np.uint16)
np.lib.stride_tricks.as_strided(arr, strides=(3, 1), shape=(2, 2))

>> array([[  0, 256],
          [512,   2]], dtype=uint16)
\end{python}

\begin{figure}[H]
\begin{verbatim}
VALUE  0       1       2       3   
----- ------------------------------
BYTE   1   2   3   4   5   6   7   8
\end{verbatim}
\end{figure}

\begin{figure}[H]
\begin{equation*}
\begin{bmatrix}
1 & \xrightarrow{1B} & 2 \\
3B \downarrow \\
4 & \xrightarrow{1B} & 5 \\
\end{bmatrix}
\end{equation*}
\end{figure}

\subsection{Getting Value Sizes}

To get sizes of each \verb+dtype+, call \verb+ar.itemsize+ for the size in \textbf{Bytes}. To ensure that your strides is correctly factored, we just multiply every value with the item size.

\begin{python}
arr = np.asarray([[0, 1], [2, 3]], dtype=np.uint16)
np.lib.stride_tricks.as_strided(arr,
                                strides=arr.itemsize * np.asarray([2, 1]),
                                shape=(2, 2))
np.lib.stride_tricks.as_strided(arr,
                                strides=[4, 2], shape=(2, 2))
\end{python}

The last 2 function calls are equivalent

\newpage
\section{3D Striding}

To avoid clutter, we'll omit the Byte row, assume it's the same as the value

\begin{verbatim}
VALUE 000 001 002 003 004 005 ... 020 021 022 023
\end{verbatim}

Let's say we want to wrap this in a 2x3x4 cube matrix.

\begin{equation*}
\begin{bmatrix}
0 & 1 & 2 \\
3 & 4 & 5 \\
\end{bmatrix}
\begin{bmatrix}
6 & 7 & 8 \\
9 & 10 & 11 \\
\end{bmatrix}
\begin{bmatrix}
12 & 13 & 14 \\
15 & 16 & 17 \\
\end{bmatrix}
\begin{bmatrix}
18 & 19 & 20 \\
21 & 22 & 23 \\
\end{bmatrix}
\end{equation*}

\begin{figure}[H]
\begin{alignat*}{2}
&\begin{bmatrix}
0 & \xrightarrow{1B} & 1 &\xrightarrow{1B} & 2 \\
3B \downarrow \\
3 & \xrightarrow{1B} & 4 &\xrightarrow{1B} & 5 \\
\end{bmatrix}
\xrightarrow{6B} 
&&\begin{bmatrix}
6 & \xrightarrow{1B} & 7 &\xrightarrow{1B} & 8 \\
3B \downarrow \\
9 & \xrightarrow{1B} & 10 &\xrightarrow{1B} & 11 \\
\end{bmatrix}\\
\xrightarrow{6B} 
&\begin{bmatrix}
12 & \xrightarrow{1B} & 13 &\xrightarrow{1B} & 14 \\
3B \downarrow \\
15 & \xrightarrow{1B} & 16 &\xrightarrow{1B} & 17 \\
\end{bmatrix} 
\xrightarrow{6B} 
&&\begin{bmatrix}
18 & \xrightarrow{1B} & 19 &\xrightarrow{1B} & 20 \\
3B \downarrow \\
21 & \xrightarrow{1B} & 22 &\xrightarrow{1B} & 23 \\
\end{bmatrix}
\end{alignat*}
\end{figure}

\begin{shaded}
Dim 2 Strides by 6,
Dim 1 Strides by 3,
Dim 0 Strides by 1.

We'll use the following mathematical equations to explain this
\end{shaded}
\begin{align*}
D_0&:(Stride=1, Shape=3)\\
D_1&:(Stride=3, Shape=2)\\
D_2&:(Stride=6, Shape=4)\\
\end{align*}

\begin{python}
arr = np.arange(0, 24, dtype=np.uint8)
np.lib.stride_tricks.as_strided(arr, strides=(6, 3, 1), shape=(4, 2, 3))

>> array([[[ 0,  1,  2],
           [ 3,  4,  5]],
          [[ 6,  7,  8],
           [ 9, 10, 11]],
          [[12, 13, 14],
           [15, 16, 17]],
          [[18, 19, 20],
           [21, 22, 23]]], dtype=uint8)
\end{python}

\newpage

\subsection{Overlapping Windows on 1D}

Let's say we want overlapping windows

\begin{equation*}
\begin{bmatrix}
0 & 1 & 2 \\
3 & 4 & 5 \\
\end{bmatrix}
\begin{bmatrix}
3 & 4 & 5 \\
6 & 7 & 8 \\
\end{bmatrix}
\begin{bmatrix}
6 & 7 & 8 \\
9 & 10 & 11 \\
\end{bmatrix}
\begin{bmatrix}
9 & 10 & 11 \\
12 & 13 & 14 \\
\end{bmatrix}
\begin{bmatrix}
12 & 13 & 14 \\
15 & 16 & 17 \\
\end{bmatrix}
\begin{bmatrix}
15 & 16 & 17 \\
18 & 19 & 20 \\
\end{bmatrix}
\begin{bmatrix}
18 & 19 & 20 \\
21 & 22 & 23 \\
\end{bmatrix}
\end{equation*}

\begin{figure}[H]
\begin{alignat*}{2}
&\begin{bmatrix}
0 & \xrightarrow{1B} & 1 &\xrightarrow{1B} & 2 \\
3B \downarrow \\
3 & \xrightarrow{1B} & 4 &\xrightarrow{1B} & 5 \\
\end{bmatrix}
%
\xrightarrow{3B} 
&&\begin{bmatrix}
3 & \xrightarrow{1B} & 4 &\xrightarrow{1B} & 5 \\
3B \downarrow \\
6 & \xrightarrow{1B} & 7 &\xrightarrow{1B} & 8 \\
\end{bmatrix} \\
%
\xrightarrow{3B} 
&\begin{bmatrix}
6 & \xrightarrow{1B} & 7 &\xrightarrow{1B} & 8 \\
3B \downarrow \\
9 & \xrightarrow{1B} & 10 &\xrightarrow{1B} & 11 \\
\end{bmatrix}
%
\xrightarrow{3B} 
&&\begin{bmatrix}
9 & \xrightarrow{1B} & 10 &\xrightarrow{1B} & 11 \\
3B \downarrow \\
12 & \xrightarrow{1B} & 13 &\xrightarrow{1B} & 14 \\
\end{bmatrix} \\
%
\xrightarrow{3B} 
&\begin{bmatrix}
12 & \xrightarrow{1B} & 13 &\xrightarrow{1B} & 14 \\
3B \downarrow \\
15 & \xrightarrow{1B} & 16 &\xrightarrow{1B} & 17 \\
\end{bmatrix}
%
\xrightarrow{3B} 
&&\begin{bmatrix}
15 & \xrightarrow{1B} & 16 &\xrightarrow{1B} & 17 \\
3B \downarrow \\
18 & \xrightarrow{1B} & 19 &\xrightarrow{1B} & 20 \\
\end{bmatrix} \\
%
\xrightarrow{3B} 
&\begin{bmatrix}
18 & \xrightarrow{1B} & 19 &\xrightarrow{1B} & 20 \\
3B \downarrow \\
21 & \xrightarrow{1B} & 22 &\xrightarrow{1B} & 23 \\
\end{bmatrix}
\end{alignat*}
\end{figure}

\begin{align*}
D_0&:(Stride=1, Shape=3)\\
D_1&:(Stride=3, Shape=2)\\
D_2&:(Stride=3, Shape=7)\\
\end{align*}

\begin{python}
arr = np.arange(0, 24, dtype=np.uint8)
np.lib.stride_tricks.as_strided(arr, strides=(3, 3, 1), shape=(7, 2, 3))

>> array([[[ 0,  1,  2],
       	   [ 3,  4,  5]],
          [[ 3,  4,  5],
           [ 6,  7,  8]],
          [[ 6,  7,  8],
           [ 9, 10, 11]],
          [[ 9, 10, 11],
           [12, 13, 14]],
          [[12, 13, 14],
           [15, 16, 17]],
          [[15, 16, 17],
           [18, 19, 20]],
          [[18, 19, 20],
           [21, 22, 23]]], dtype=uint8)
\end{python}

\subsection{Overlapping Windows on 2D}

Let's try another to understand better. This is much more complicated, but if we do the diagram, it's slightly clearer.

\begin{equation*}
\begin{bmatrix}
0 & 1 & 2 \\
1 & 2 & 3 \\
2 & 3 & 4 \\
3 & 4 & 5 \\
\end{bmatrix}
\begin{bmatrix}
3 & 4 & 5 \\
4 & 5 & 6 \\
5 & 6 & 7 \\
6 & 7 & 8 \\
\end{bmatrix}
\begin{bmatrix}
...
\end{bmatrix}
\begin{bmatrix}
15 & 16 & 17 \\
16 & 17 & 18 \\
17 & 18 & 19 \\
18 & 19 & 20 \\
\end{bmatrix}
\begin{bmatrix}
18 & 19 & 20 \\
19 & 20 & 21 \\
20 & 21 & 22 \\
21 & 22 & 23 \\
\end{bmatrix}
\end{equation*}

\begin{figure}[H]
\begin{alignat*}{2}
&\begin{bmatrix}
0 & \xrightarrow{1B} & 1 &\xrightarrow{1B} & 2 \\
1B \downarrow \\
1 & \xrightarrow{1B} & 2 &\xrightarrow{1B} & 3 \\
1B \downarrow \\
2 & \xrightarrow{1B} & 3 &\xrightarrow{1B} & 4 \\
1B \downarrow \\
3 & \xrightarrow{1B} & 4 &\xrightarrow{1B} & 5 \\
\end{bmatrix}
%
\xrightarrow{3B}
&&\begin{bmatrix}
3 & \xrightarrow{1B} & 4 &\xrightarrow{1B} & 5 \\
1B \downarrow \\
4 & \xrightarrow{1B} & 5 &\xrightarrow{1B} & 6 \\
1B \downarrow \\
5 & \xrightarrow{1B} & 6 &\xrightarrow{1B} & 7 \\
1B \downarrow \\
6 & \xrightarrow{1B} & 7 &\xrightarrow{1B} & 8 \\
\end{bmatrix}
%
\xrightarrow{3B} ... \\ 
... \xrightarrow{3B}
&\begin{bmatrix}
15 & \xrightarrow{1B} & 16 &\xrightarrow{1B} & 17 \\
1B \downarrow \\
16 & \xrightarrow{1B} & 17 &\xrightarrow{1B} & 18 \\
1B \downarrow \\
17 & \xrightarrow{1B} & 18 &\xrightarrow{1B} & 19 \\
1B \downarrow \\
18 & \xrightarrow{1B} & 19 &\xrightarrow{1B} & 20 \\
\end{bmatrix}
%
\xrightarrow{3B}
&&\begin{bmatrix}
18 & \xrightarrow{1B} & 19 &\xrightarrow{1B} & 20 \\
1B \downarrow \\
19 & \xrightarrow{1B} & 20 &\xrightarrow{1B} & 21 \\
1B \downarrow \\
20 & \xrightarrow{1B} & 21 &\xrightarrow{1B} & 22 \\
1B \downarrow \\
21 & \xrightarrow{1B} & 22 &\xrightarrow{1B} & 23 \\
\end{bmatrix}
\end{alignat*}
\end{figure}

\begin{align*}
D_0&:(Stride=1, Shape=3)\\
D_1&:(Stride=1, Shape=4)\\
D_2&:(Stride=3, Shape=7)\\
\end{align*}

\begin{python}
arr = np.arange(0, 24, dtype=np.uint8)
np.lib.stride_tricks.as_strided(arr, strides=(3, 1, 1), shape=(7, 4, 3))

>> array([[[ 0,  1,  2],
           [ 1,  2,  3],
           [ 2,  3,  4],
           [ 3,  4,  5]],
          [[ 3,  4,  5],
           [ 4,  5,  6],
           [ 5,  6,  7],
           [ 6,  7,  8]], 
           ...
          [[15, 16, 17],
           [16, 17, 18],
           [17, 18, 19],
           [18, 19, 20]],
          [[18, 19, 20],
           [19, 20, 21],
           [20, 21, 22],
           [21, 22, 23]]], dtype=uint8)
\end{python}

As long as you use the template, I don't think you should run into any issues, however, you're free to reinterpret it.

\section{References}

\href{https://stackoverflow.com/a/59322185}{(Stack Overflow) How can I simply calculate the rolling/moving variance of a time series in python?}

\href{https://stackoverflow.com/a/53099870}{(Stack Overflow) How to understand numpy strides for layman? (1)}

\href{https://stackoverflow.com/a/53224165}{(Stack Overflow) How to understand numpy strides for layman? (2)}

\end{document}